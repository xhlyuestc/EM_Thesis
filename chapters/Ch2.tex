% !Mode:: "TeX:UTF-8"

\chapter{语音呼叫系统的相关技术}

\section{嵌入式系统概述}
\subsection{嵌入式系统的定义}
嵌{\cf}入式{\cf}系统{\cf}是以{\cf}应用{\cf}为中{\cf}心、{\cf}以计{\cf}算机{\cf}技术{\cf}为基{\cf}础、{\cf}软件{\cf}硬件{\cf}可裁{\cf}剪、{\cf}适应{\cf}应用{\cf}系统{\cf}对功{\cf}能、{\cf}可靠{\cf}性、{\cf}成本{\cf}、体{\cf}积、{\cf}功耗{\cf}严格{\cf}要求{\cf}的专{\cf}用计{\cf}算机{\cf}系统{\cf}。嵌{\cf}入式{\cf}系统{\cf}是将{\cf}先进{\cf}的计{\cf}算机{\cf}技术{\cf}、半{\cf}导体{\cf}技术{\cf}和电{\cf}子技{\cf}术与{\cf}各个{\cf}行业{\cf}的具{\cf}体应{\cf}用相{\cf}结合{\cf}后的{\cf}产物。

嵌入式系统包含有计算机,但又不是通用计算机的计算机应用系统。通用计算机与嵌入式系统对比见表~\ref{Table_PC_vs_Embedded}
\threelinetable[htbp]{Table_PC_vs_Embedded}{1.0\textwidth}{c|p{0.4\textwidth}<{\centering}|p{0.4\textwidth}<{\centering}}{通用计算机与嵌入式系统对比}
{特征&	通用计算机&	嵌入式系统\\
}{
形式和类型
&
\begin{itemize}
\item 看{\cf}得见{\cf}的计算机。
\item 按其{\cf}体系{\cf}结构{\cf}、运{\cf}算速{\cf}度和{\cf}结构{\cf}规模{\cf}等因{\cf}素分{\cf}为大{\cf}、中{\cf}、小{\cf}型机{\cf}和微机。
\end{itemize}
&
\begin{itemize}
\item 看{\cf}不见{\cf}的计算机。
\item 形{\cf}式多{\cf}样,{\cf}应用{\cf}领域{\cf}广泛{\cf},按{\cf}应用{\cf}来分。
\end{itemize}
\\
\hline
组成
&
\begin{itemize}
\item 通{\cf}用处{\cf}理器{\cf}、标{\cf}准总{\cf}线和{\cf}外设。
\item 软{\cf}件和{\cf}硬件{\cf}相对{\cf}独立。
\end{itemize}
&
\begin{itemize}
\item 面{\cf}向应{\cf}用的{\cf}嵌入{\cf}式微{\cf}处理{\cf}器,{\cf}总线{\cf}和外{\cf}部接{\cf}口多{\cf}集成{\cf}在处{\cf}理器{\cf}内部。
\item 软{\cf}件与{\cf}硬件{\cf}是紧{\cf}密集{\cf}成在{\cf}一起的。
\end{itemize}
\\
\hline
开发方式
&
\begin{itemize}
\item 开{\cf}发平{\cf}台和{\cf}运行{\cf}平台{\cf}都是{\cf}通用{\cf}计算{\cf}机。
\end{itemize}
&
\begin{itemize}
\item 采{\cf}用交{\cf}叉开{\cf}发方{\cf}式,{\cf}开发{\cf}平台{\cf}一般{\cf}是通{\cf}用计{\cf}算机{\cf},运{\cf}行平{\cf}台是{\cf}嵌入{\cf}式系{\cf}统。
\end{itemize}
\\
\hline
二{\cf}次开{\cf}发性
&
\begin{itemize}
\item 应{\cf}用程{\cf}序可{\cf}重新{\cf}编制
\end{itemize}
&
\begin{itemize}
\item 一{\cf}般不{\cf}能再编程
\end{itemize}
\\
}{
}

\subsection{嵌入式系统的组成}
嵌{\cf}入式{\cf}系统{\cf}一般{\cf}由嵌{\cf}入式{\cf}硬件{\cf}和软{\cf}件组{\cf}成。{\cf}硬件{\cf}以微{\cf}处理{\cf}器为{\cf}核心{\cf}集成{\cf}存储{\cf}器和{\cf}系统{\cf}专用{\cf}的输{\cf}入/{\cf}输出{\cf}设备{\cf};软{\cf}件包{\cf}括:{\cf}初始{\cf}化代{\cf}码及{\cf}驱动{\cf}、嵌{\cf}入式{\cf}操作{\cf}系统{\cf}和应{\cf}用程{\cf}序等{\cf},这{\cf}些软{\cf}件有{\cf}机地{\cf}结合{\cf}在一{\cf}起,{\cf}形成{\cf}系统{\cf}特定{\cf}的一{\cf}体化{\cf}软件。典型的嵌入式系统框架如图~\ref{embedded_frame.png}~所示。
\pic[htbp]{典型的嵌入式系统框架}{width=0.5\textwidth}{embedded_frame.png}

\subsection{嵌入式系统的特点}
嵌{\cf}入式{\cf}系统{\cf}具有{\cf}以下特点\citeup{wumin2007,zhangchao2007,zhoulei2007,songdan2008}:
\begin{enumerate}
\item 嵌{\cf}入式{\cf}系统{\cf}通常{\cf}是形{\cf}式多{\cf}样、{\cf}面向{\cf}特定{\cf}应用{\cf}的

一{\cf}般用{\cf}于特{\cf}定的{\cf}任务{\cf},其{\cf}硬件{\cf}和软{\cf}件都{\cf}必须{\cf}高效{\cf}率地{\cf}设计{\cf},量{\cf}体裁{\cf}衣、{\cf}去除{\cf}冗余{\cf},而{\cf}通用{\cf}计算{\cf}机则{\cf}是一{\cf}个通{\cf}用的{\cf}计算{\cf}平台{\cf}。 {\cf}它通{\cf}常都{\cf}具有{\cf}低功{\cf}耗、{\cf}体积{\cf}小、{\cf}集成{\cf}度高{\cf}等特{\cf}点,{\cf}能够{\cf}把通{\cf}用微{\cf}处理{\cf}器中{\cf}许多{\cf}由板{\cf}卡完{\cf}成的{\cf}任务{\cf}集成{\cf}在芯{\cf}片内{\cf}部。{\cf}嵌入{\cf}式软{\cf}件是{\cf}应用{\cf}程序{\cf}和操{\cf}作系{\cf}统两{\cf}种软{\cf}件的{\cf}一体{\cf}化程序。

\item 嵌{\cf}入式{\cf}系统{\cf}得到{\cf}多种{\cf}类型{\cf}的处{\cf}理器{\cf}和处{\cf}理器{\cf}体系{\cf}结构{\cf}的支持

通{\cf}用计{\cf}算机{\cf}采用{\cf}少数{\cf}的处{\cf}理器{\cf}类型{\cf}和体{\cf}系结{\cf}构,{\cf}而且{\cf}主要{\cf}掌握{\cf}在少{\cf}数大{\cf}公司{\cf}手里{\cf}。 {\cf}嵌入{\cf}式系{\cf}统可{\cf}采用{\cf}多种{\cf}类型{\cf}的处{\cf}理器{\cf}和处{\cf}理器{\cf}体系{\cf}结构。

在{\cf}嵌入{\cf}式微{\cf}处理{\cf}器产{\cf}业链{\cf}上,{\cf}IP{\cf}设计{\cf}、面{\cf}向应{\cf}用的{\cf}特定{\cf}嵌入{\cf}式微{\cf}处理{\cf}器的{\cf}设计{\cf}、芯{\cf}片的{\cf}制造{\cf}已相{\cf}成巨{\cf}大的{\cf}产业{\cf}。大{\cf}家分{\cf}工协{\cf}作,{\cf}形成{\cf}多赢{\cf}模式{\cf}。有{\cf}上千{\cf}种的{\cf}嵌入{\cf}式微{\cf}处理{\cf}器和{\cf}几十{\cf}种嵌{\cf}入式{\cf}微处{\cf}理器{\cf}体系{\cf}结构{\cf}可以{\cf}选择{\cf}。

\item 嵌入{\cf}式系{\cf}统通{\cf}常极{\cf}其关{\cf}注成{\cf}本;

嵌{\cf}入式{\cf}系统{\cf}通常{\cf}需要{\cf}注意{\cf}的成{\cf}本是{\cf}系统{\cf}成本{\cf},特{\cf}别是{\cf}量大{\cf}的消{\cf}费类{\cf}数字{\cf}化产{\cf}品,{\cf}其成{\cf}本是{\cf}产品{\cf}竞争{\cf}的关{\cf}键因{\cf}素之{\cf}一。{\cf}嵌入{\cf}式的{\cf}系统{\cf}成本{\cf}包括:
\begin{enumerate}
\item 一次{\cf}性的{\cf}开发(\newacronym[description=一次性的开发]{NRE}{NRE}{Non-Recurring Engineering}\acrlong{NRE},\acrshort{NRE})成本;
\item 产{\cf}品成{\cf}本:{\cf}硬件{\cf}物料(\newacronym[description=物料清单]{BOM}{BOM}{Bill of Material}\acrlong{BOM},\acrshort{BOM})成{\cf}本、{\cf}外壳{\cf}包装{\cf}和软{\cf}件版{\cf}税等;
\item 批{\cf}量产{\cf}品的{\cf}总体{\cf}成本{\cf}=N{\cf}RE{\cf}成本{\cf}+每{\cf}个产{\cf}品成{\cf}本*{\cf}产品{\cf}总量{\cf};
\item 每{\cf}个产{\cf}品的{\cf}最后{\cf}成本{\cf}=总{\cf}体成{\cf}本/{\cf}产品{\cf}总量{\cf}=N{\cf}RE{\cf}成本{\cf}/产{\cf}品总{\cf}量+{\cf}每个{\cf}产品{\cf}成本。
\end{enumerate}

\item 嵌{\cf}入式{\cf}系统{\cf}有实{\cf}时性{\cf}和可{\cf}靠性{\cf}的要{\cf}求

一{\cf}方面{\cf}大多{\cf}数实{\cf}时系{\cf}统都{\cf}是嵌{\cf}入式{\cf}系统{\cf},另{\cf}一方{\cf}面嵌{\cf}入式{\cf}系统{\cf}多数{\cf}有实{\cf}时性{\cf}的要{\cf}求,{\cf}软件{\cf}一般{\cf}是固{\cf}化运{\cf}行或{\cf}直接{\cf}加载{\cf}到内{\cf}存中{\cf}运行{\cf},具{\cf}有快{\cf}速启{\cf}动的{\cf}功能{\cf}。并{\cf}对实{\cf}时的{\cf}强度{\cf}要求{\cf}各不{\cf}一样{\cf},可{\cf}分为{\cf}硬实{\cf}时和{\cf}软实{\cf}时。{\cf}嵌入{\cf}式系{\cf}统一{\cf}般要{\cf}求具{\cf}有出{\cf}错处{\cf}理和{\cf}自动{\cf}复位{\cf}功能{\cf},特{\cf}别是{\cf}对于{\cf}一些{\cf}在极{\cf}端环{\cf}境下{\cf}运行{\cf}的嵌{\cf}入式{\cf}系统{\cf}而言{\cf},其{\cf}可靠{\cf}性设{\cf}计尤{\cf}其重{\cf}要。{\cf}在大{\cf}多数{\cf}嵌入{\cf}式系{\cf}统的{\cf}软件{\cf}中一{\cf}般都{\cf}包括{\cf}一些{\cf}机制{\cf},比{\cf}如硬{\cf}件的{\cf}看门{\cf}狗定{\cf}时器{\cf},软{\cf}件的{\cf}内存{\cf}保护{\cf}和重{\cf}启动{\cf}机制{\cf}。

\item 嵌{\cf}入式{\cf}系统{\cf}使用{\cf}的操{\cf}作系{\cf}统一{\cf}般是{\cf}适应{\cf}多种{\cf}处理{\cf}器、{\cf}可剪{\cf}裁、{\cf}轻量{\cf}型、{\cf}实时{\cf}可靠{\cf}、可{\cf}固化{\cf}的嵌{\cf}入式{\cf}操作{\cf}系统{\cf}

由{\cf}于嵌{\cf}入式{\cf}系统{\cf}应用{\cf}的特{\cf}点,{\cf}像嵌{\cf}入式{\cf}微处{\cf}理器{\cf}一样{\cf},嵌{\cf}入式{\cf}操作{\cf}系统{\cf}也是{\cf}多姿{\cf}多彩{\cf}的。{\cf}大多{\cf}数商{\cf}业嵌{\cf}入式{\cf}操作{\cf}系统{\cf}可同{\cf}时支{\cf}持不{\cf}同种{\cf}类的{\cf}嵌入{\cf}式微{\cf}处理{\cf}器。{\cf}可根{\cf}据应{\cf}用的{\cf}情况{\cf}进行{\cf}剪裁{\cf}、配{\cf}置。{\cf}嵌入{\cf}式操{\cf}作系{\cf}统规{\cf}模小{\cf},所{\cf}需的{\cf}资源{\cf}有限{\cf}如内{\cf}核规{\cf}模在{\cf}几十{\cf}KB{\cf},能{\cf}与应{\cf}用软{\cf}件一{\cf}样固{\cf}化运{\cf}行。{\cf}一般{\cf}包括{\cf}一个{\cf}实时{\cf}内核{\cf},其{\cf}调度{\cf}算法{\cf}一般{\cf}采用{\cf}基于{\cf}优先{\cf}级的{\cf}可抢{\cf}占的{\cf}调度{\cf}算法{\cf}。高{\cf}可靠{\cf}嵌入{\cf}式操{\cf}作系{\cf}统:{\cf}时、{\cf}空、{\cf}数据{\cf}隔离。

\item 嵌{\cf}入式{\cf}系统{\cf}开发{\cf}需要{\cf}专门{\cf}工具{\cf}和特{\cf}殊方{\cf}法

多{\cf}数嵌{\cf}入式{\cf}系统{\cf}开发{\cf}意味{\cf}着软{\cf}件与{\cf}硬件{\cf}的并{\cf}行设{\cf}计和{\cf}开发{\cf},其{\cf}开发{\cf}过程{\cf}一般{\cf}分为{\cf}几个{\cf}阶段{\cf}:产{\cf}品定{\cf}义、{\cf}软件{\cf}与硬{\cf}件设{\cf}计与{\cf}实现{\cf}、软{\cf}件与{\cf}硬件{\cf}集成{\cf}、产{\cf}品测{\cf}试与{\cf}发布{\cf}、维{\cf}护与{\cf}升级{\cf}。由{\cf}于嵌{\cf}入式{\cf}系统{\cf}资源{\cf}有限{\cf},一{\cf}般不{\cf}具备{\cf}自主{\cf}开发{\cf}能力{\cf},产{\cf}品发{\cf}布后{\cf}用户{\cf}通常{\cf}也不{\cf}能对{\cf}其中{\cf}的软{\cf}件进{\cf}行修{\cf}改,{\cf}必须{\cf}有一{\cf}套专{\cf}门的{\cf}开发{\cf}环境{\cf}。该{\cf}开发{\cf}环境{\cf}包括{\cf}专门{\cf}的开{\cf}发工{\cf}具({\cf}包括{\cf}设计{\cf}、编{\cf}译、{\cf}调试{\cf}、测{\cf}试等{\cf}工具{\cf}),{\cf}采用{\cf}交叉{\cf}开发{\cf}的方{\cf}式进{\cf}行,{\cf}交叉开发环境如图~\ref{cross_develop.png}~所示。
\pic[htbp]{嵌入式系统的交叉开发环境}{width=0.6\textwidth}{cross_develop.png}
\end{enumerate}

\section{面向对象的程序设计技术}
面{\cf}向对{\cf}象的{\cf}程序{\cf}设计(\newacronym[description=面向对象的程序设计]{OOP}{OOP}{Object Oriented Programming}\acrlong{OOP},\acrshort{OOP})是{\cf}一种{\cf}广泛{\cf}使用{\cf}的计{\cf}算机{\cf}编程{\cf}架构{\cf}。面{\cf}向对{\cf}象的{\cf}程序{\cf}设计{\cf}的一{\cf}条最{\cf}基本{\cf}的原{\cf}则就{\cf}是计{\cf}算机{\cf}程序{\cf}是由{\cf}单个{\cf}的能{\cf}够起{\cf}到子{\cf}程序{\cf}作用{\cf}的单{\cf}元或{\cf}对象{\cf}组合{\cf}而成的。

\subsection{对象}
面{\cf}向对{\cf}象程{\cf}序设{\cf}计既{\cf}是一{\cf}种程{\cf}序设{\cf}计范{\cf}型,{\cf}同时{\cf}也是{\cf}一种{\cf}程序{\cf}开发{\cf}的方{\cf}法。{\cf}对象{\cf}指的{\cf}是类{\cf}的实例。\acrshort{OOP}将对{\cf}象作{\cf}为程{\cf}序的{\cf}基本{\cf}单元{\cf},将{\cf}程序{\cf}和数{\cf}据封{\cf}装其{\cf}中,{\cf}以提{\cf}高软{\cf}件的{\cf}重用{\cf}性、{\cf}灵活{\cf}性和{\cf}扩展{\cf}性。

一{\cf}个具{\cf}体对{\cf}象属{\cf}性的{\cf}值被{\cf}称作{\cf}它的{\cf} “状{\cf}态” {\cf}。({\cf}系统{\cf}给对{\cf}象分{\cf}配内{\cf}存空{\cf}间,{\cf}而不{\cf}会给{\cf}类分{\cf}配内{\cf}存空{\cf}间。{\cf}这很{\cf}好理{\cf}解,{\cf}类是{\cf}抽象{\cf}的系{\cf}统不{\cf}可能{\cf}给抽{\cf}象的{\cf}东西{\cf}分配{\cf}空间{\cf},而{\cf}对象{\cf}则是{\cf}具体{\cf}的。

\subsection{组件}
即{\cf}数据{\cf}和功{\cf}能一{\cf}起在{\cf}运行{\cf}着的{\cf}计算{\cf}机程{\cf}序中{\cf}形成{\cf}单元{\cf},组{\cf}件在{\cf}面向{\cf}对象{\cf}的程{\cf}序设{\cf}计所{\cf}设计{\cf}的计{\cf}算机{\cf}程序{\cf}中是{\cf}模块{\cf}化和{\cf}结构{\cf}化的{\cf}基础。

\subsection{封装}
也{\cf}叫做{\cf}信息{\cf}封装{\cf},即{\cf}确保{\cf}组件{\cf}不会{\cf}以某{\cf}种不{\cf}可预{\cf}期的{\cf}方式{\cf}改变{\cf}其它{\cf}组件{\cf}的内{\cf}部运{\cf}行状{\cf}态;{\cf}只有{\cf}在那{\cf}些提{\cf}供了{\cf}内部{\cf}运行{\cf}状态{\cf}改变{\cf}方法{\cf}的组{\cf}件中{\cf},才{\cf}可以{\cf}访问{\cf}其内{\cf}部运{\cf}行状{\cf}态。{\cf}每类{\cf}组件{\cf}都提{\cf}供了{\cf}一个{\cf}与其{\cf}它组{\cf}件进{\cf}行联{\cf}系的{\cf}接口{\cf},并{\cf}且规{\cf}定了{\cf}其它{\cf}组件{\cf}对其{\cf}进行{\cf}调用{\cf}的方{\cf}法。

\subsection{消息传递}
一{\cf}个对{\cf}象通{\cf}过接{\cf}受消{\cf}息、{\cf}处理{\cf}消息{\cf}、传{\cf}出消{\cf}息或{\cf}使用{\cf}其他{\cf}类的{\cf}方法{\cf}来实{\cf}现一{\cf}定功{\cf}能,{\cf}这叫{\cf}做消{\cf}息传{\cf}递机制(Message Passing)。

面{\cf}向对{\cf}象程{\cf}序设{\cf}计可{\cf}以看{\cf}作一{\cf}种在{\cf}程序{\cf}中包{\cf}含各{\cf}种独{\cf}立而{\cf}又互{\cf}相调{\cf}用的{\cf}对象{\cf}的思{\cf}想,{\cf}这与{\cf}传统{\cf}的思{\cf}想刚{\cf}好相{\cf}反:{\cf}传统{\cf}的程{\cf}序设{\cf}计主{\cf}张将{\cf}程序{\cf}看作{\cf}一系{\cf}列函{\cf}数的{\cf}集合{\cf},或{\cf}者直{\cf}接就{\cf}是一{\cf}系列{\cf}对电{\cf}脑下{\cf}达的{\cf}指令{\cf}。为{\cf}了实{\cf}现整{\cf}体运{\cf}算,{\cf}面向{\cf}对象{\cf}程序{\cf}设计{\cf}中的{\cf}每一{\cf}个对{\cf}象都{\cf}应该{\cf}能够{\cf}接受{\cf}数据{\cf}、处{\cf}理数{\cf}据并{\cf}将数{\cf}据传{\cf}达给{\cf}其它{\cf}对象{\cf},因{\cf}此它{\cf}们都{\cf}可以{\cf}被看{\cf}作一{\cf}个小{\cf}型的{\cf} “机{\cf}器” {\cf},即{\cf}对象{\cf}。

\subsection{\acrshort{OOP}的主要特性}
为什么病房语音呼叫系统的开发要选用面向对象的程序设计呢?因为面向对象的程序设计实现了软件工程的三个主要目标:重用性、灵活性和扩展性。
 
图4-1	程序设计的层次
面向对象的程序设计具备下列特性\citeup{litingting2009}:

抽{\cf}象性{\cf}——{\cf}即程{\cf}序有{\cf}能力{\cf}忽略{\cf}正在{\cf}处理{\cf}中的{\cf}某些{\cf}信息{\cf}的某{\cf}些方{\cf}面,{\cf}即对{\cf}信息{\cf}的主{\cf}要方{\cf}面进{\cf}行关{\cf}注的{\cf}能力。

多{\cf}态性{\cf}——{\cf}即组{\cf}件的{\cf}引用{\cf}以及{\cf}类集{\cf}会涉{\cf}及到{\cf}许多{\cf}其它{\cf}不同{\cf}类型{\cf}的组{\cf}件,{\cf}而且{\cf}引用{\cf}组件{\cf}所产{\cf}生的{\cf}结果{\cf}必须{\cf}依据{\cf}实际{\cf}调用{\cf}的类型。

继{\cf}承性{\cf}——{\cf}即允{\cf}许在{\cf}现存{\cf}的组{\cf}件基{\cf}础之{\cf}上创{\cf}建子{\cf}类组{\cf}件,{\cf}这统{\cf}一并{\cf}且增{\cf}强了{\cf}程序{\cf}的多{\cf}态性{\cf}和封{\cf}装性{\cf}。简{\cf}单地{\cf}说来{\cf}就是{\cf}使用{\cf}类来{\cf}对组{\cf}件进{\cf}行划{\cf}分,{\cf}而且{\cf}还可{\cf}以定{\cf}义新{\cf}类作{\cf}为现{\cf}存类{\cf}的扩{\cf}展,{\cf}于是{\cf}这样{\cf}就可{\cf}以将{\cf}类组{\cf}织成{\cf}树形{\cf}结构{\cf}或者{\cf}网状{\cf}结构{\cf},同{\cf}时这{\cf}也体{\cf}现了{\cf}动作{\cf}的通{\cf}用性。

正是因为面向对象的程序设计具有以上的优越特点,才使得它满足病房语音呼叫系统的开发需求,方便程序接口的设计。

\newacronym[description=传输控制协议/因特网互联协议]{TCPIP}{TCP/IP}{Transmission Control Protocol/Internet Protocol}
\section{\acrshort{TCPIP}协议}
\acrshort{TCPIP}是Transmission Control Protocol/Internet Protocol的简写,中{\cf}译名{\cf}为传{\cf}输控{\cf}制协{\cf}议/{\cf}因特{\cf}网互{\cf}联协{\cf}议,{\cf}又名{\cf}网络{\cf}通讯{\cf}协议{\cf},是{\cf}In{\cf}te{\cf}rn{\cf}et{\cf}最基{\cf}本的{\cf}协议{\cf}、I{\cf}nt{\cf}er{\cf}ne{\cf}t国{\cf}际互{\cf}联网{\cf}络的{\cf}基础{\cf},由{\cf}网络{\cf}层的{\cf}IP{\cf}协议{\cf}和传{\cf}输层{\cf}的T{\cf}CP{\cf}协议组成。\acrshort{TCPIP}定义了电子设备如何连入因特网,以及数据如何在它们之间传输的标准。协{\cf}议采{\cf}用了{\cf}4层{\cf}的层{\cf}级结{\cf}构,{\cf}每一{\cf}层都{\cf}呼叫{\cf}它的{\cf}下一{\cf}层所{\cf}提供{\cf}的协{\cf}议来{\cf}完成{\cf}自己{\cf}的需{\cf}求。{\cf}通俗{\cf}而言{\cf}:T{\cf}CP{\cf}负责{\cf}发现{\cf}传输{\cf}的问{\cf}题,{\cf}一有{\cf}问题{\cf}就发{\cf}出信{\cf}号,{\cf}要求{\cf}重新{\cf}传输{\cf},直{\cf}到所{\cf}有数{\cf}据安{\cf}全正{\cf}确地{\cf}传输{\cf}到目{\cf}的地{\cf}。而{\cf}IP{\cf}是给{\cf}因特{\cf}网的{\cf}每一{\cf}台联{\cf}网设{\cf}备规{\cf}定一{\cf}个地址\citeup{liuzhe2013,mahuijuan2010}。

\subsection{\acrshort{TCPIP}协议的层次}
\acrshort{TCPIP}协{\cf}议不{\cf}是T{\cf}CP{\cf}和I{\cf}P这{\cf}两个{\cf}协议{\cf}的合{\cf}称,{\cf}而是{\cf}指因{\cf}特网整个\acrshort{TCPIP}协议族。

从{\cf}协议{\cf}分层{\cf}模型{\cf}方面{\cf}来讲{\cf},\acrshort{TCPIP}由{\cf}四个{\cf}层次{\cf}组成{\cf}:网{\cf}络接{\cf}口层{\cf}、网{\cf}络层{\cf}、传{\cf}输层{\cf}、应{\cf}用层。

\acrshort{TCPIP}协{\cf}议并{\cf}不完{\cf}全符{\cf}合OSI(\newacronym[description=传统的开放式系统互连]{OSI}{OSI}{Open System Interconnect},\acrlong{OSI})的{\cf}七层{\cf}参考{\cf}模型{\cf},\acrshort{OSI}是{\cf}传统{\cf}的开{\cf}放式{\cf}系统{\cf}互连{\cf}参考{\cf}模型{\cf},是{\cf}一种{\cf}通信{\cf}协议{\cf}的7{\cf}层抽{\cf}象的{\cf}参考{\cf}模型{\cf},其{\cf}中每{\cf}一层{\cf}执行{\cf}某一{\cf}特定{\cf}任务{\cf}。该{\cf}模型{\cf}的目{\cf}的是{\cf}使各{\cf}种硬{\cf}件在{\cf}相同{\cf}的层{\cf}次上{\cf}相互{\cf}通信{\cf}。这{\cf}7层{\cf}是:{\cf}物理{\cf}层、{\cf}数据{\cf}链路{\cf}层({\cf}网络{\cf}接口{\cf}层){\cf}、网{\cf}络层{\cf}(网{\cf}络层{\cf})、{\cf}传输{\cf}层({\cf}传输{\cf}层){\cf}、会{\cf}话层{\cf}、表{\cf}示层{\cf}和应{\cf}用层{\cf}(应{\cf}用层{\cf})。而\acrshort{TCPIP}通{\cf}讯协{\cf}议采{\cf}用了{\cf}4层{\cf}的层{\cf}级结{\cf}构,{\cf}每一{\cf}层都{\cf}呼叫{\cf}它的{\cf}下一{\cf}层所{\cf}提供{\cf}的网{\cf}络来{\cf}完成{\cf}自己{\cf}的需{\cf}求。{\cf}由于{\cf}AR{\cf}PA{\cf}NE{\cf}T的{\cf}设计{\cf}者注{\cf}重的{\cf}是网{\cf}络互{\cf}联,{\cf}允许{\cf}通信{\cf}子网{\cf}(网{\cf}络接{\cf}口层{\cf})采{\cf}用已{\cf}有的{\cf}或是{\cf}将来{\cf}有的{\cf}各种{\cf}协议{\cf},所{\cf}以这{\cf}个层{\cf}次中{\cf}没有{\cf}提供{\cf}专门{\cf}的协{\cf}议。{\cf}实际{\cf}上,\acrshort{TCPIP}协{\cf}议可{\cf}以通{\cf}过网{\cf}络接{\cf}口层{\cf}连接{\cf}到任{\cf}何网{\cf}络上{\cf},例{\cf}如X{\cf}.2{\cf}5交{\cf}换网{\cf}或I{\cf}EE{\cf}E8{\cf}02局域网。

\subsection{IP}
IP{\cf}层接{\cf}收由{\cf}更低{\cf}层({\cf}网络{\cf}接口{\cf}层例{\cf}如以{\cf}太网{\cf}设备{\cf}驱动{\cf}程序{\cf})发{\cf}来的{\cf}数据{\cf}包,{\cf}并把{\cf}该数{\cf}据包{\cf}发送{\cf}到更{\cf}高层{\cf}--{\cf}-T{\cf}CP{\cf}或U{\cf}DP{\cf}层;{\cf}相反{\cf},I{\cf}P层{\cf}也把{\cf}从T{\cf}CP{\cf}或U{\cf}DP{\cf}层接{\cf}收来{\cf}的数{\cf}据包{\cf}传送{\cf}到更{\cf}低层{\cf}。I{\cf}P数{\cf}据包{\cf}是不{\cf}可靠{\cf}的,{\cf}因为{\cf}IP{\cf}并没{\cf}有做{\cf}任何{\cf}事情{\cf}来确{\cf}认数{\cf}据包{\cf}是否{\cf}按顺{\cf}序发{\cf}送的{\cf}或者{\cf}有没{\cf}有被{\cf}破坏{\cf},I{\cf}P数{\cf}据包{\cf}中含{\cf}有发{\cf}送它{\cf}的主{\cf}机的{\cf}地址{\cf}(源{\cf}地址{\cf})和{\cf}接收{\cf}它的{\cf}主机{\cf}的地{\cf}址({\cf}目的{\cf}地址{\cf})\citeup{chenhongbo2006}。

高{\cf}层的{\cf}TC{\cf}P和{\cf}UD{\cf}P服{\cf}务在{\cf}接收{\cf}数据{\cf}包时{\cf},通{\cf}常假{\cf}设包{\cf}中的{\cf}源地{\cf}址是{\cf}有效{\cf}的。{\cf}也可{\cf}以这{\cf}样说{\cf},I{\cf}P地{\cf}址形{\cf}成了{\cf}许多{\cf}服务{\cf}的认{\cf}证基{\cf}础,{\cf}这些{\cf}服务{\cf}相信{\cf}数据{\cf}包是{\cf}从一{\cf}个有{\cf}效的{\cf}主机{\cf}发送{\cf}来的{\cf}。I{\cf}P确{\cf}认包{\cf}含一{\cf}个选{\cf}项,{\cf}叫作{\cf}IP{\cf} s{\cf}ou{\cf}rc{\cf}e {\cf}ro{\cf}ut{\cf}in{\cf}g,{\cf}可以{\cf}用来{\cf}指定{\cf}一条{\cf}源地{\cf}址和{\cf}目的{\cf}地址{\cf}之间{\cf}的直{\cf}接路{\cf}径。{\cf}对于{\cf}一些{\cf}TC{\cf}P和{\cf}UD{\cf}P的{\cf}服务{\cf}来说{\cf},使{\cf}用了{\cf}该选{\cf}项的{\cf}IP{\cf}包好{\cf}像是{\cf}从路{\cf}径上{\cf}的最{\cf}后一{\cf}个系{\cf}统传{\cf}递过{\cf}来的{\cf},而{\cf}不是{\cf}来自{\cf}于它{\cf}的真{\cf}实地{\cf}点。{\cf}这个{\cf}选项{\cf}是为{\cf}了测{\cf}试而{\cf}存在{\cf}的,{\cf}说明{\cf}了它{\cf}可以{\cf}被用{\cf}来欺{\cf}骗系{\cf}统来{\cf}进行{\cf}平常{\cf}是被{\cf}禁止{\cf}的连{\cf}接。{\cf}那么{\cf},许{\cf}多依{\cf}靠I{\cf}P源{\cf}地址{\cf}做确{\cf}认的{\cf}服务{\cf}将产{\cf}生问{\cf}题并{\cf}且会{\cf}被非{\cf}法入{\cf}侵\citeup{chenhongbo2006}。

\subsection{TCP}
TCP{\cf}是面{\cf}向连{\cf}接的{\cf}通信{\cf}协议{\cf},通{\cf}过三{\cf}次握{\cf}手建{\cf}立连{\cf}接,{\cf}通讯{\cf}完成{\cf}时要{\cf}拆除{\cf}连接{\cf},由{\cf}于T{\cf}CP{\cf}是面{\cf}向连{\cf}接的{\cf}所以{\cf}只能{\cf}用于{\cf}端到{\cf}端的{\cf}通讯。

TCP{\cf}提供{\cf}的是{\cf}一种{\cf}可靠{\cf}的数{\cf}据流{\cf}服务{\cf},采{\cf}用“{\cf}带重{\cf}传的{\cf}肯定{\cf}确认{\cf}”技{\cf}术来{\cf}实现{\cf}传输{\cf}的可{\cf}靠性{\cf}。T{\cf}CP{\cf}还采{\cf}用一{\cf}种称{\cf}为“{\cf}滑动{\cf}窗口{\cf}”的{\cf}方式{\cf}进行{\cf}流量{\cf}控制{\cf},所{\cf}谓窗{\cf}口实{\cf}际表{\cf}示接{\cf}收能{\cf}力,{\cf}用以{\cf}限制{\cf}发送{\cf}方的{\cf}发送速度。

如{\cf}果I{\cf}P数{\cf}据包{\cf}中有{\cf}已经{\cf}封好{\cf}的T{\cf}CP{\cf}数据{\cf}包,{\cf}那么{\cf}IP{\cf}将把{\cf}它们{\cf}向‘{\cf}上’{\cf}传送{\cf}到T{\cf}CP{\cf}层。{\cf}TC{\cf}P将{\cf}包排{\cf}序并{\cf}进行{\cf}错误{\cf}检查{\cf},同{\cf}时实{\cf}现虚{\cf}电路{\cf}间的{\cf}连接{\cf}。T{\cf}CP{\cf}数据{\cf}包中{\cf}包括{\cf}序号{\cf}和确{\cf}认,{\cf}所以{\cf}未按{\cf}照顺{\cf}序收{\cf}到的{\cf}包可{\cf}以被{\cf}排序{\cf},而{\cf}损坏{\cf}的包{\cf}可以{\cf}被重传\citeup{chenhongbo2006,shenfengnian2009}

TCP{\cf}将它{\cf}的信{\cf}息送{\cf}到更{\cf}高层{\cf}的应{\cf}用程{\cf}序,{\cf}例如{\cf}Te{\cf}ln{\cf}et{\cf}的服{\cf}务程{\cf}序和{\cf}客户{\cf}程序{\cf}。应{\cf}用程{\cf}序轮{\cf}流将{\cf}信息{\cf}送回{\cf}TC{\cf}P层{\cf},T{\cf}CP{\cf}层便{\cf}将它{\cf}们向{\cf}下传{\cf}送到{\cf}IP{\cf}层,{\cf}设备{\cf}驱动{\cf}程序{\cf}和物{\cf}理介{\cf}质,{\cf}最后{\cf}到接收方。

面{\cf}向连{\cf}接的{\cf}服务{\cf}(例{\cf}如T{\cf}el{\cf}ne{\cf}t、{\cf}FT{\cf}P、{\cf}rl{\cf}og{\cf}in{\cf}、X{\cf} W{\cf}in{\cf}do{\cf}ws{\cf}和S{\cf}MT{\cf}P){\cf}需要{\cf}高度{\cf}的可{\cf}靠性{\cf},所{\cf}以它{\cf}们使{\cf}用了{\cf}TCP。\newacronym[description=域名解析系统]{DNS}{DNS}{Domain Name System}\acrshort{DNS}(\acrlong{DNS},域名解析系统)在{\cf}某些{\cf}情况{\cf}下使{\cf}用T{\cf}CP{\cf}(发{\cf}送和{\cf}接收{\cf}域名{\cf}数据{\cf}库){\cf},但{\cf}使用{\cf}UD{\cf}P传{\cf}送有{\cf}关单{\cf}个主{\cf}机的{\cf}信息。

\subsection{UDP}
UDP{\cf}是面{\cf}向无{\cf}连接{\cf}的通{\cf}讯协{\cf}议,{\cf}UD{\cf}P数{\cf}据包{\cf}括目{\cf}的端{\cf}口号{\cf}和源{\cf}端口{\cf}号信{\cf}息,{\cf}由于{\cf}通讯{\cf}不需{\cf}要连{\cf}接,{\cf}所以{\cf}可以{\cf}实现{\cf}广播{\cf}发送。

UDP{\cf}通讯{\cf}时不{\cf}需要{\cf}接收{\cf}方确{\cf}认,{\cf}属于{\cf}不可{\cf}靠的{\cf}传输{\cf},可{\cf}能会{\cf}出现{\cf}丢包{\cf}现象{\cf},实{\cf}际应{\cf}用中{\cf}要求{\cf}程序{\cf}员编{\cf}程验证。

UDP{\cf}与T{\cf}CP{\cf}位于{\cf}同一{\cf}层,{\cf}但它{\cf}不管{\cf}数据{\cf}包的{\cf}顺序{\cf}、错{\cf}误或{\cf}重发{\cf}。因{\cf}此,{\cf}UD{\cf}P不{\cf}被应{\cf}用于{\cf}那些{\cf}使用{\cf}虚电{\cf}路的{\cf}面向{\cf}连接{\cf}的服{\cf}务,{\cf}UD{\cf}P主{\cf}要用{\cf}于那{\cf}些面{\cf}向查{\cf}询-{\cf}--{\cf}应答{\cf}的服{\cf}务,{\cf}例如{\cf}NF{\cf}S。{\cf}相对{\cf}于F{\cf}TP{\cf}或T{\cf}el{\cf}ne{\cf}t,{\cf}这些{\cf}服务{\cf}需要{\cf}交换{\cf}的信{\cf}息量{\cf}较小{\cf}。使{\cf}用U{\cf}DP{\cf}的服{\cf}务包括\newacronym[description=网络时间协议]{NTP}{NTP}{Network Time Protocol}\acrshort{NTP}(\acrlong{NTP},网络时间协议)和\acrshort{DNS}(\acrshort{DNS}也使用TCP)。

欺{\cf}骗U{\cf}DP{\cf}包比{\cf}欺骗{\cf}TC{\cf}P包{\cf}更容{\cf}易,{\cf}因为{\cf}UD{\cf}P没{\cf}有建{\cf}立初{\cf}始化{\cf}连接{\cf}(也{\cf}可以{\cf}称为{\cf}握手{\cf},因{\cf}为在{\cf}两个{\cf}系统{\cf}间没{\cf}有虚{\cf}电路{\cf}),{\cf}也就{\cf}是说{\cf},与{\cf}UD{\cf}P相{\cf}关的{\cf}服务{\cf}面临{\cf}着更{\cf}大的{\cf}危险\citeup{chenhongbo2006,shenfengnian2009}。

%\subsection{ICMP}
%ICMP与IP位于同一层,它被用来传送IP的控制信息。它主要是用来提供有关通向目的地址的路径信息。ICMP的‘Redirect’信息通知主机通向其他系统的更准确的路径,而‘Unreachable’信息则指出路径有问题。另外,如果路径不可用了,ICMP可以使TCP连接‘体面地’终止。PING是最常用的基于ICMP的服务。

\section{数据库技术}	
上{\cf}位机{\cf}系统{\cf}需要{\cf}通过{\cf}计算{\cf}机通{\cf}信接{\cf}口采{\cf}集所{\cf}有下{\cf}位机{\cf}用户{\cf}的用{\cf}电数{\cf}据,{\cf}这些{\cf}数据{\cf}随时{\cf}间积{\cf}累会{\cf}变得{\cf}很庞{\cf}大,{\cf}好在{\cf}它们{\cf}都是{\cf}有固{\cf}定格{\cf}式的{\cf}数据{\cf},使{\cf}用数{\cf}据库{\cf}对其{\cf}进行{\cf}管理{\cf}便自{\cf}然而{\cf}然地{\cf}成了{\cf}最佳{\cf}选择\citeup{liyanting2012}。

数{\cf}据库{\cf}技术{\cf}是信{\cf}息系{\cf}统的{\cf}一个{\cf}核心{\cf}技术{\cf}。是{\cf}一种{\cf}计算{\cf}机辅{\cf}助管{\cf}理数{\cf}据的{\cf}方法{\cf},它{\cf}研究{\cf}如何{\cf}组织{\cf}和存{\cf}储数{\cf}据,{\cf}如何{\cf}高效{\cf}地获{\cf}取和{\cf}处理{\cf}数据{\cf}。是{\cf}通过{\cf}研究{\cf}数据{\cf}库的{\cf}结构{\cf}、存{\cf}储、{\cf}设计{\cf}、管{\cf}理以{\cf}及应{\cf}用的{\cf}基本{\cf}理论{\cf}和实{\cf}现方{\cf}法,{\cf}并利{\cf}用这{\cf}些理{\cf}论来{\cf}实现{\cf}对数{\cf}据库{\cf}中的{\cf}数据{\cf}进行{\cf}处理{\cf}、分{\cf}析和{\cf}理解{\cf}的技{\cf}术。{\cf}即:{\cf}数据{\cf}库技{\cf}术是{\cf}研究{\cf}、管{\cf}理和{\cf}应用{\cf}数据{\cf}库的{\cf}一门{\cf}软件{\cf}科学\citeup{luoying2012}。

数{\cf}据库{\cf}技术{\cf}是现{\cf}代信{\cf}息科{\cf}学与{\cf}技术{\cf}的重{\cf}要组{\cf}成部{\cf}分,{\cf}是计{\cf}算机{\cf}数据{\cf}处理{\cf}与信{\cf}息管{\cf}理系{\cf}统的{\cf}核心{\cf}。数{\cf}据库{\cf}技术{\cf}研究{\cf}和解{\cf}决了{\cf}计算{\cf}机信{\cf}息处{\cf}理过{\cf}程中{\cf}大量{\cf}数据{\cf}有效{\cf}地组{\cf}织和{\cf}存储{\cf}的问{\cf}题,{\cf}在数{\cf}据库{\cf}系统{\cf}中减{\cf}少数{\cf}据存{\cf}储冗{\cf}余、{\cf}实现{\cf}数据{\cf}共享{\cf}、保{\cf}障数{\cf}据安{\cf}全以{\cf}及高{\cf}效地{\cf}检索{\cf}数据{\cf}和处{\cf}理数据。

数{\cf}据库{\cf}技术{\cf}研究{\cf}和管{\cf}理的{\cf}对象{\cf}是数{\cf}据,{\cf}所以{\cf}数据{\cf}库技{\cf}术所{\cf}涉及{\cf}的具{\cf}体内{\cf}容主{\cf}要包{\cf}括:{\cf}通过{\cf}对数{\cf}据的{\cf}统一{\cf}组织{\cf}和管{\cf}理,{\cf}按照{\cf}指定{\cf}的结{\cf}构建{\cf}立相{\cf}应的{\cf}数据{\cf}库和{\cf}数据{\cf}仓库{\cf};利{\cf}用数{\cf}据库{\cf}管理{\cf}系统{\cf}和数{\cf}据挖{\cf}掘系{\cf}统设{\cf}计出{\cf}能够{\cf}实现{\cf}对数{\cf}据库{\cf}中的{\cf}数据{\cf}进行{\cf}添加{\cf}、修{\cf}改、{\cf}删除{\cf}、处{\cf}理、{\cf}分析{\cf}、理{\cf}解、{\cf}报表{\cf}和打{\cf}印等{\cf}多种{\cf}功能{\cf}的数{\cf}据管{\cf}理和{\cf}数据{\cf}挖掘{\cf}应用{\cf}系统{\cf};并{\cf}利用{\cf}应用{\cf}管理{\cf}系统{\cf}最终{\cf}实现{\cf}对数{\cf}据的{\cf}处理{\cf}、分{\cf}析和{\cf}理解\citeup{weidahai2011}。

\section{本章小结}
本章主要对病房语音呼叫系统中需要涉及到的主要相关技术及其相关概念进行了简要介绍和概括:嵌入式系统、面向对象的程序设计技术、\acrshort{TCPIP}协议、数据库技术等。
