% !Mode:: "TeX:UTF-8"

\chapter{语音呼叫系统的相关技术}

\section{嵌入式系统概述}
\subsection{嵌入式系统的定义}
嵌入式系统是以应用为中心、以计算机技术为基础、软件硬件可裁剪、适应应用系统对功能、可靠性、成本、体积、功耗严格要求的专用计算机系统。嵌入式系统是将先进的计算机技术、半导体技术和电子技术与各个行业的具体应用相结合后的产物。

嵌入式系统包含有计算机,但又不是通用计算机的计算机应用系统。通用计算机与嵌入式系统对比见表~\ref{Table_PC_vs_Embedded}
\threelinetable[htbp]{Table_PC_vs_Embedded}{1.0\textwidth}{c|p{0.4\textwidth}<{\centering}|p{0.4\textwidth}<{\centering}}{通用计算机与嵌入式系统对比}
{特征&	通用计算机&	嵌入式系统\\
}{
形式和类型
&
\begin{itemize}
\item 看得见的计算机。
\item 按其体系结构、运算速度和结构规模等因素分为大、中、小型机和微机。
\end{itemize}
&
\begin{itemize}
\item 看不见的计算机。
\item 形式多样,应用领域广泛,按应用来分。
\end{itemize}
\\
\hline
组成
&
\begin{itemize}
\item 通用处理器、标准总线和外设。
\item 软件和硬件相对独立。
\end{itemize}
&
\begin{itemize}
\item 面向应用的嵌入式微处理器,总线和外部接口多集成在处理器内部。
\item 软件与硬件是紧密集成在一起的。
\end{itemize}
\\
\hline
开发方式
&
\begin{itemize}
\item 开发平台和运行平台都是通用计算机。
\end{itemize}
&
\begin{itemize}
\item 采用交叉开发方式,开发平台一般是通用计算机,运行平台是嵌入式系统。
\end{itemize}
\\
\hline
二次开发性
&
\begin{itemize}
\item 应用程序可重新编制
\end{itemize}
&
\begin{itemize}
\item 一般不能再编程
\end{itemize}
\\
}{
}

\subsection{嵌入式系统的组成}
嵌入式系统一般由嵌入式硬件和软件组成。硬件以微处理器为核心集成存储器和系统专用的输入/输出设备;软件包括:初始化代码及驱动、嵌入式操作系统和应用程序等,这些软件有机地结合在一起,形成系统特定的一体化软件。

\subsection{嵌入式系统的特点}
嵌入式系统具有以下特点:
\begin{enumerate}
\item 嵌入式系统通常是形式多样、面向特定应用的

一般用于特定的任务,其硬件和软件都必须高效率地设计,量体裁衣、去除冗余,而通用计算机则是一个通用的计算平台。 它通常都具有低功耗、体积小、集成度高等特点,能够把通用微处理器中许多由板卡完成的任务集成在芯片内部。嵌入式软件是应用程序和操作系统两种软件的一体化程序。

\item 嵌入式系统得到多种类型的处理器和处理器体系结构的支持

通用计算机采用少数的处理器类型和体系结构,而且主要掌握在少数大公司手里。 嵌入式系统可采用多种类型的处理器和处理器体系结构。
在嵌入式微处理器产业链上,IP设计、面向应用的特定嵌入式微处理器的设计、芯片的制造已相成巨大的产业。大家分工协作,形成多赢模式。有上千种的嵌入式微处理器和几十种嵌入式微处理器体系结构可以选择。

\item 嵌入式系统通常极其关注成本;

嵌入式系统通常需要注意的成本是系统成本,特别是量大的消费类数字化产品,其成本是产品竞争的关键因素之一。嵌入式的系统成本包括:
\begin{itemize}
\item 一次性的开发(\newacronym[description=一次性的开发]{NRE}{NRE}{Non-Recurring Engineering}\acrlong{NRE},\acrshort{NRE})成本;
\item 产品成本:硬件物料(\newacronym[description=物料清单]{BOM}{BOM}{Bill of Material}\acrlong{BOM},\acrshort{BOM})成本、外壳包装和软件版税等;
\item 批量产品的总体成本=NRE成本+每个产品成本*产品总量;
\item 每个产品的最后成本=总体成本/产品总量=NRE成本/产品总量+每个产品成本。
\end{itemize}

\item 嵌入式系统有实时性和可靠性的要求

一方面大多数实时系统都是嵌入式系统,另一方面嵌入式系统多数有实时性的要求,软件一般是固化运行或直接加载到内存中运行,具有快速启动的功能。并对实时的强度要求各不一样,可分为硬实时和软实时。嵌入式系统一般要求具有出错处理和自动复位功能,特别是对于一些在极端环境下运行的嵌入式系统而言,其可靠性设计尤其重要。在大多数嵌入式系统的软件中一般都包括一些机制,比如硬件的看门狗定时器,软件的内存保护和重启动机制。

\item 嵌入式系统使用的操作系统一般是适应多种处理器、可剪裁、轻量型、实时可靠、可固化的嵌入式操作系统

由于嵌入式系统应用的特点,像嵌入式微处理器一样,嵌入式操作系统也是多姿多彩的。大多数商业嵌入式操作系统可同时支持不同种类的嵌入式微处理器。可根据应用的情况进行剪裁、配置。嵌入式操作系统规模小,所需的资源有限如内核规模在几十KB,能与应用软件一样固化运行。一般包括一个实时内核,其调度算法一般采用基于优先级的可抢占的调度算法。高可靠嵌入式操作系统:时、空、数据隔离。

\item 嵌入式系统开发需要专门工具和特殊方法

多数嵌入式系统开发意味着软件与硬件的并行设计和开发,其开发过程一般分为几个阶段:产品定义、软件与硬件设计与实现、软件与硬件集成、产品测试与发布、维护与升级。由于嵌入式系统资源有限,一般不具备自主开发能力,产品发布后用户通常也不能对其中的软件进行修改,必须有一套专门的开发环境。该开发环境包括专门的开发工具(包括设计、编译、调试、测试等工具),采用交叉开发的方式进行,交叉开发环境如图~\ref{cross_develop.png}所示。
\end{enumerate}

\section{面向对象的程序设计技术}
面向对象的程序设计(\newacronym[description=面向对象的程序设计]{OOP}{OOP}{Object Oriented Programming}\acrlong{OOP},\acrshort{OOP})是一种广泛使用的计算机编程架构。面向对象的程序设计的一条最基本的原则就是计算机程序是由单个的能够起到子程序作用的单元或对象组合而成的。

\subsection{对象}
面向对象程序设计既是一种程序设计范型,同时也是一种程序开发的方法。对象指的是类的实例。\acrshort{OOP}将对象作为程序的基本单元,将程序和数据封装其中,以提高软件的重用性、灵活性和扩展性。[1]

一个具体对象属性的值被称作它的“状态”。(系统给对象分配内存空间,而不会给类分配内存空间。这很好理解,类是抽象的系统不可能给抽象的东西分配空间,而对象则是具体的。

\subsection{组件}
即数据和功能一起在运行着的计算机程序中形成单元,组件在面向对象的程序设计所设计的计算机程序中是模块化和结构化的基础。

\subsection{封装}
也叫做信息封装,即确保组件不会以某种不可预期的方式改变其它组件的内部运行状态;只有在那些提供了内部运行状态改变方法的组件中,才可以访问其内部运行状态。每类组件都提供了一个与其它组件进行联系的接口,并且规定了其它组件对其进行调用的方法。

\subsection{消息传递}
一个对象通过接受消息、处理消息、传出消息或使用其他类的方法来实现一定功能,这叫做消息传递机制(Message Passing)。

面向对象程序设计可以看作一种在程序中包含各种独立而又互相调用的对象的思想,这与传统的思想刚好相反:传统的程序设计主张将程序看作一系列函数的集合,或者直接就是一系列对电脑下达的指令。为了实现整体运算,面向对象程序设计中的每一个对象都应该能够接受数据、处理数据并将数据传达给其它对象,因此它们都可以被看作一个小型的“机器”,即对象。

\subsection{\acrshort{OOP}的主要特性}
为什么病房语音呼叫系统的开发要选用面向对象的程序设计呢?因为面向对象的程序设计实现了软件工程的三个主要目标:重用性、灵活性和扩展性。
 
图4-1	程序设计的层次
面向对象的程序设计具备下列特性:

抽象性——即程序有能力忽略正在处理中的某些信息的某些方面,即对信息的主要方面进行关注的能力。

多态性——即组件的引用以及类集会涉及到许多其它不同类型的组件,而且引用组件所产生的结果必须依据实际调用的类型。

继承性——即允许在现存的组件基础之上创建子类组件,这统一并且增强了程序的多态性和封装性。简单地说来就是使用类来对组件进行划分,而且还可以定义新类作为现存类的扩展,于是这样就可以将类组织成树形结构或者网状结构,同时这也体现了动作的通用性。

正是因为面向对象的程序设计具有以上的优越特点,才使得它满足病房语音呼叫系统的开发需求,方便程序接口的设计。

\newacronym[description=传输控制协议/因特网互联协议]{TCPIP}{TCP/IP}{Transmission Control Protocol/Internet Protocol}
\section{\acrshort{TCPIP}协议}
\acrshort{TCPIP}是Transmission Control Protocol/Internet Protocol的简写,中译名为传输控制协议/因特网互联协议,又名网络通讯协议,是Internet最基本的协议、Internet国际互联网络的基础,由网络层的IP协议和传输层的TCP协议组成。\acrshort{TCPIP}定义了电子设备如何连入因特网,以及数据如何在它们之间传输的标准。协议采用了4层的层级结构,每一层都呼叫它的下一层所提供的协议来完成自己的需求。通俗而言:TCP负责发现传输的问题,一有问题就发出信号,要求重新传输,直到所有数据安全正确地传输到目的地。而IP是给因特网的每一台联网设备规定一个地址。

\subsection{\acrshort{TCPIP}协议的层次}
\acrshort{TCPIP}协议不是TCP和IP这两个协议的合称,而是指因特网整个\acrshort{TCPIP}协议族。

从协议分层模型方面来讲,\acrshort{TCPIP}由四个层次组成:网络接口层、网络层、传输层、应用层。

\acrshort{TCPIP}协议并不完全符合OSI(\newacronym[description=传统的开放式系统互连]{OSI}{OSI}{Open System Interconnect},\acrlong{OSI})的七层参考模型,\acrshort{OSI}是传统的开放式系统互连参考模型,是一种通信协议的7层抽象的参考模型,其中每一层执行某一特定任务。该模型的目的是使各种硬件在相同的层次上相互通信。这7层是:物理层、数据链路层(网络接口层)、网络层(网络层)、传输层(传输层)、会话层、表示层和应用层(应用层)。而\acrshort{TCPIP}通讯协议采用了4层的层级结构,每一层都呼叫它的下一层所提供的网络来完成自己的需求。由于ARPANET的设计者注重的是网络互联,允许通信子网(网络接口层)采用已有的或是将来有的各种协议,所以这个层次中没有提供专门的协议。实际上,\acrshort{TCPIP}协议可以通过网络接口层连接到任何网络上,例如X.25交换网或IEEE802局域网。

\subsection{IP}
IP层接收由更低层(网络接口层例如以太网设备驱动程序)发来的数据包,并把该数据包发送到更高层---TCP或UDP层;相反,IP层也把从TCP或UDP层接收来的数据包传送到更低层。IP数据包是不可靠的,因为IP并没有做任何事情来确认数据包是否按顺序发送的或者有没有被破坏,IP数据包中含有发送它的主机的地址(源地址)和接收它的主机的地址(目的地址)。

高层的TCP和UDP服务在接收数据包时,通常假设包中的源地址是有效的。也可以这样说,IP地址形成了许多服务的认证基础,这些服务相信数据包是从一个有效的主机发送来的。IP确认包含一个选项,叫作IP source routing,可以用来指定一条源地址和目的地址之间的直接路径。对于一些TCP和UDP的服务来说,使用了该选项的IP包好像是从路径上的最后一个系统传递过来的,而不是来自于它的真实地点。这个选项是为了测试而存在的,说明了它可以被用来欺骗系统来进行平常是被禁止的连接。那么,许多依靠IP源地址做确认的服务将产生问题并且会被非法入侵。

\subsection{TCP}
TCP是面向连接的通信协议,通过三次握手建立连接,通讯完成时要拆除连接,由于TCP是面向连接的所以只能用于端到端的通讯。

TCP提供的是一种可靠的数据流服务,采用“带重传的肯定确认”技术来实现传输的可靠性。TCP还采用一种称为“滑动窗口”的方式进行流量控制,所谓窗口实际表示接收能力,用以限制发送方的发送速度。

如果IP数据包中有已经封好的TCP数据包,那么IP将把它们向‘上’传送到TCP层。TCP将包排序并进行错误检查,同时实现虚电路间的连接。TCP数据包中包括序号和确认,所以未按照顺序收到的包可以被排序,而损坏的包可以被重传。

TCP将它的信息送到更高层的应用程序,例如Telnet的服务程序和客户程序。应用程序轮流将信息送回TCP层,TCP层便将它们向下传送到IP层,设备驱动程序和物理介质,最后到接收方。

面向连接的服务(例如Telnet、FTP、rlogin、X Windows和SMTP)需要高度的可靠性,所以它们使用了TCP。\newacronym[description=域名解析系统]{DNS}{DNS}{Domain Name System}\acrshort{DNS}(\acrlong{DNS},域名解析系统)在某些情况下使用TCP(发送和接收域名数据库),但使用UDP传送有关单个主机的信息。

\subsection{UDP}
UDP是面向无连接的通讯协议,UDP数据包括目的端口号和源端口号信息,由于通讯不需要连接,所以可以实现广播发送。

UDP通讯时不需要接收方确认,属于不可靠的传输,可能会出现丢包现象,实际应用中要求程序员编程验证。

UDP与TCP位于同一层,但它不管数据包的顺序、错误或重发。因此,UDP不被应用于那些使用虚电路的面向连接的服务,UDP主要用于那些面向查询---应答的服务,例如NFS。相对于FTP或Telnet,这些服务需要交换的信息量较小。使用UDP的服务包括\newacronym[description=网络时间协议]{NTP}{NTP}{Network Time Protocol}\acrshort{NTP}(\acrlong{NTP},网络时间协议)和\acrshort{DNS}(\acrshort{DNS}也使用TCP)。

欺骗UDP包比欺骗TCP包更容易,因为UDP没有建立初始化连接(也可以称为握手,因为在两个系统间没有虚电路),也就是说,与UDP相关的服务面临着更大的危险。

%\subsection{ICMP}
%ICMP与IP位于同一层,它被用来传送IP的控制信息。它主要是用来提供有关通向目的地址的路径信息。ICMP的‘Redirect’信息通知主机通向其他系统的更准确的路径,而‘Unreachable’信息则指出路径有问题。另外,如果路径不可用了,ICMP可以使TCP连接‘体面地’终止。PING是最常用的基于ICMP的服务。

\section{数据库技术}	
上位机系统需要通过计算机通信接口采集所有下位机用户的用电数据,这些数据随时间积累会变得很庞大,好在它们都是有固定格式的数据,使用数据库对其进行管理便自然而然地成了最佳选择。

数据库技术是信息系统的一个核心技术。是一种计算机辅助管理数据的方法,它研究如何组织和存储数据,如何高效地获取和处理数据。是通过研究数据库的结构、存储、设计、管理以及应用的基本理论和实现方法,并利用这些理论来实现对数据库中的数据进行处理、分析和理解的技术。即:数据库技术是研究、管理和应用数据库的一门软件科学。

数据库技术是现代信息科学与技术的重要组成部分,是计算机数据处理与信息管理系统的核心。数据库技术研究和解决了计算机信息处理过程中大量数据有效地组织和存储的问题,在数据库系统中减少数据存储冗余、实现数据共享、保障数据安全以及高效地检索数据和处理数据。

数据库技术研究和管理的对象是数据,所以数据库技术所涉及的具体内容主要包括:通过对数据的统一组织和管理,按照指定的结构建立相应的数据库和数据仓库;利用数据库管理系统和数据挖掘系统设计出能够实现对数据库中的数据进行添加、修改、删除、处理、分析、理解、报表和打印等多种功能的数据管理和数据挖掘应用系统;并利用应用管理系统最终实现对数据的处理、分析和理解。

\section{本章小结}
本章主要对病房语音呼叫系统中需要涉及到的主要相关技术及其相关概念进行了简要介绍和概括:嵌入式系统、面向对象的程序设计技术、\acrshort{TCPIP}协议、数据库技术等。