% !Mode:: "TeX:UTF-8"

\chapter{系统的部署与联合测试}
\section{系统软件安装}
系统部署和测试之前,首先应该给目标计算机安装系统上位机软件,因为后续的设备参数配置、联合调试都倚赖系统的上位机软件。安装方法和步骤跟常规软件的安装完全相同,在此不做赘述。

\section{系统网络部署}
系统部署时需要给系统终端提供外部~+5~V~直流电源,因此首先需要铺设统一供电的电缆或者给每个终端设备提供一个单独的小型化AC220V/DC+5V电源适配器。

由于系统使用\acrshort{WiFi}作为网络接入方式,所以省去了大量铺设通信电缆的成本和人力,系统网络部署时只需要确保终端设备的工作环境(病房)有良好的\acrshort{WiFi}覆盖,并保证终端设备天线连接可靠,无金属障碍物遮挡即可。\acrshort{WiFi}接入点(AP,通常为路由器)的设置参考~\ref{sec:device_config}~节。

\section{设备参数配置}\label{sec:device_config}
系统终端设备供电和\acrshort{WiFi}覆盖都满足后,便可对设备进行参数配置,以便设备顺利接入网络。用户可以使用串口或者\acrshort{WiFi}对设备进行参数配置。由于初始情况下,设备只能按默认参数进行\acrshort{WiFi}接入,若果使用\acrshort{WiFi}进行配置,那么应将接入点按照设备默认参数进行设置,即使用“router{\_}ssid”作为路由器的SSID,并且使用默认的用户名“admin”和密码“admin”,终端设备才可以自动进行连接,否则应该通过串口将\acrshort{WiFi}用户名和密码下发给终端设备,随后设备会保存用户名和密码、自动重启并以指定的用户名和密码接入\acrshort{WiFi}。此时,运行上位机软件,选择【设备管理】$\rightarrow$【参数配置】,弹出对话窗口如图~\ref{device_config.png}~所示。
\pic[H]{设备参数配置对话框}{}{device_config.png}

下发给中断设备的参数还包括设备的IP获取方式:DHCP或静态IP,DHCP由路由器动态分配IP,IP地址不固定,如果希望每个终端设备有固定IP,则应选择静态IP方式,此时还应该为其指定IP地址、子网掩码、默认网关和DNS地址。设备的参数配置还包括给设备指定设备名称、设备定时向服务器汇报自身情况(例如电源电量、有无系统异常等)的时间间隔、指定服务器IP(对于固定IP服务器)或服务器URL(对于使用域名解析的服务器)、服务器端口和服务器名称。这些参数配置完成后,设备首先将参数存储到FLASH中并向配置端返回参数配置成功的确认包,随后自动重启并按新的参数初始化系统,自动接入\acrshort{WiFi}并连接服务器,定时向服务器发送心跳汇报包。
\pic[htbp]{服务器收到设备状态汇报}{}{report.png}

参数配置完毕后,应观察设备的指示灯是否正常指示系统的工作状态,是否能够正常连接服务器并发送汇报包,如图~\ref{report.png}。若不正常应该检查原因或者重新进行参数配置。待以上项目都正常以后按下语音呼叫按钮,服务器端应该收到语音呼叫请求,并正确显示呼叫设备的名称、IP地址等相关信息,如图~\ref{call.png}~所示。确认以上信息均为正确后点击确认接受呼叫请求,测试能否正常地建立语音连接和通话质量。最后结束语音通话,检查终端设备是否正常关闭了语音输出,设备指示灯恢复到待机状态并定时向服务器汇报自身状态。
\pic[htbp]{服务器收到语音呼叫请求}{}{call.png}

参数配置到此完毕。

\section{系统联合测试}
系统软件安装、系统网络部署、设备参数配置都完成后便可开始系统联合测试,此项测试只旨在检验系统在实际运行环境中多设备共存、服务器多业务并发执行、多客户端并发访问、数据库频繁大量的数据操作等情况下的稳定性。系统联合测试按如下方法进行:
\begin{enumerate}
\item 根据实际需求选择部署的终端设备数量,这决定了系统网络的规模(为保证测试的有效性,建议设备节点数在100以上),完成设备参数配置和网络部署,并保证网络参数分配合理,同一局域网中绝对不允许有两个或两个以上相同设备IP和设备名称。
\item 开启无线接入点并保证网络畅通,运行系统上位机软件并启动系统服务器,最后给所有的终端设备供电。此时服务器应该能收到每一个终端设备周期性的心跳汇报包,对~24~h~(或更长时间)内每一个设备的汇报包进行统计,考察是否有长时的“心跳停止”。
\item 尽可能同时按下多个终端设备的语音呼叫按钮,考察服务端是否能同时响应这些设备的语音呼叫请求,重复测试,统计并发呼叫终端设备和服务器实际响应的语音呼叫请求情况,考察是否有无语音呼叫请求被丢失。
\item 重复发起语音呼叫请求并建立语音连接(建议每个终端设备至少重复100次),测试语音呼叫结束后终端设备是否正常关闭了语音输出,设备指示灯是否恢复到待机状态并定时向服务器汇报自身状态。
\item 其他上位机界面的操作和病房信息管理等功能测试在短时间内只能做基本功能测试,稳定性方面只能在系统的实际运行过程中注意观察,一旦有故障发生应该记录故障现象和诱因以便后续故障定位和修复。
\end{enumerate}

\section{本章小结}
本章主要介绍了系统软件安装、系统网络部署、设备参数配置以及系统联合测试的详细方法与步骤,给出了样例测试结果。经过实际测试,系统的功能完整正常,系统稳定性良好,功能和性能均达到了设计要求。