% !Mode:: "TeX:UTF-8"

\chapter{系统硬件的设计与实现}\label{CH3}
\section{系统硬件的总体方案设计}
任何通信都涉及到信道和传输媒介的问题,考虑到有线通信方式在系统部署时需要铺设大量的通信电缆,产生巨大的安装成本而且也不利于设备维护,影响外观。而目前\newacronym[description=一种基于802.11b标准的无线局域网传输协议]{WiFi}{Wi-Fi}{Wireless Fidelity}
\acrshort{WiFi}(\acrlong{WiFi})的覆盖越来越广泛,使用\acrshort{WiFi}作为病房语音呼叫系统接入网络的方式就成了不言而喻的选择。系统硬件上以嵌入式微控制器为核心,通过外设IO与各个电路或模块互联,总体方案框图如图~\ref{hw_block.pdf}所示。
\pic[htbp]{系统硬件的总体方案框图}{width=0.5\textwidth}{hw_block.pdf}

\section{系统电源设计}
考虑到病房语音呼叫的特殊性,系统硬件一般情况下使用外部直流电源供电,遇到停电等特殊情况时使用内部电池作为后备电源。外部电源供电时通过TP4056锂电池电源管理芯片给电池充电以延长电池寿命。为尽量降低功耗,系统电源芯片RT8008在常态下是关闭的,只在三种情况下才开启:(1)定时唤醒;(2)语音呼叫按钮被按下;(3)防破坏拆除按钮被断开。任何一种情况都会是RT8008的使能信号Power{\_}en有效,系统电源开启,详见第~\ref{sec:butn_pdwn}~节阐述。
\pic[htbp]{系统电源电路原理图}{width=0.8\textwidth}{power.pdf}

\section{微控制器最小系统设计}
\subsection{STM32嵌入式微控制器简介}
STM32系列嵌入式微控制器基于专为要求高性能、低成本、低功耗的嵌入式应用专门设计的ARM\reg Cortex\reg -M3内核。ARM\reg Cortex\reg -M3 处理器是行业领先的 32 位处理器,适用于具有较高确定性的实时应用,它经过专门开发,可使合作伙伴针对广泛的设备(包括微控制器、汽车车身系统、工业控制系统以及无线网络和传感器)开发高性能低成本平台。此处理器具有出色的计算性能以及对事件的优异系统响应能力,同时可应实际中对低动态和静态功率需求的挑战。此处理器配置十分灵活,从而支持广泛的实现形式(从需要内存保护和强大 trace 技术的实现形式,直至需要极小面积的成本敏感型设备)。

STM32系列嵌入式微控制器按性能分成两个不同的系列:STM32F103“增强型”系列和STM32F101“基本型”系列。增强型系列时钟频率达到72MHz,是同类产品中性能最高的产品;基本型时钟频率为36MHz,以16位产品的价格得到比16位产品大幅提升的性能,是16位产品用户的最佳选择。两个系列都内置32K到128K的闪存,不同的是SRAM的最大容量和外设接口的组合。时钟频率72MHz时,从闪存执行代码,STM32功耗36mA,是32位市场上功耗最低的产品,相当于0.5mA/MHz。

\subsection{基于STM32的嵌入式最小系统}
事实上STM32单片机最小系统是不需要外部时钟的,其内部集成了RC高速振荡器可作为系统时钟源,只不过考虑到语音采样或者还原需要比较精确的时钟控制,选择了片外晶体振荡器为STM32提供系统时钟。预留的SWD接口为STM32的编程调试接口,用官方的ST-LINK或者第三方工具J-LINK可以方便地对STM32进行在线调试或编程。一红一绿两个LED用于指示系统工作状态:通常情况下,绿色指示灯每1s闪烁一次表示系统工作正常且处于待机状态;用户按下语音呼叫按钮后红色LED闪烁表示正在呼叫中;语音呼叫接通后,绿色LED长亮表示处于语音通话状态;如果长时间不能接通,则红色LED长亮表示通信故障;语音通话结束后,回到待机状态,绿色LED每秒闪烁一次。
\pic[H]{STM32最小系统电路原理图}{width=0.9\textwidth}{mini_sys.pdf}

\section{语音输入电路}
语音输入电路如图~\ref{mic.pdf}所示,语音通过驻极话筒转化为电信号,经电容耦合进入放大电路,微弱的语音信号经过两级运算放大器从mV级别放大到约2.5 Vpp左右,送入STM32的模拟信号输入引脚,最后经其内部的ADC(\newacronym[description=模数转换器]{ADC}{ADC}{Analog-to-Digital Converter}\acrlong{ADC}, 模拟数字转换器)转化为数字语音信号。
\pic[htbp]{语音输入电路原理图}{width=0.9\textwidth}{mic.pdf}

\section{语音输出电路}
对方用户端采集和处理后的数字语音信号通过网络实时传输到接收端,接收端的STM32微控制器则将语音数据流通过其片内的DAC(\newacronym[description=数模转换器]{DAC}{DAC}{Digital-Analog Converter}\acrlong{DAC}, 数字模拟转换器))按特定的速率转换为模拟语音电信号,再经过低通滤波去除高频数字噪声、功率放大后驱动扬声器还原成声音。语音功放芯片带有使能引脚SHTN,接高电平时正常工作,接低电平时停止工作,消耗极小的电流。STM32微控制器通过PB1引脚控制其工作状态,待机时关闭语音功放,语音通话时开启语音功放。
\pic[htbp]{语音输出电路原理图}{width=0.8\textwidth}{speaker.pdf}

\section{\acrshort{WiFi}模块电路}
\acrshort{WiFi}模块WM-G-MR-08外围电路也比较简单,主要包括天线匹配电路、接收低噪声放大器(\newacronym[description=低噪声放大器]{LNA}{LNA}{Low Noise Amplifier}\acrlong{LNA}, \acrshort{LNA})的自动增益控制(\newacronym[description=自动增益控制]{AGC}{AGC}{Automatic Gain Control}\acrlong{AGC}, \acrshort{AGC})设置。\acrshort{WiFi}模块通过\newacronym[description=安全数字输入输出]{SDIO}{SDIO}{Secure Digital Input and Output}
\acrshort{SDIO}(\acrlong{SDIO},安全数字输入输出)接口与STM32微控制器通信,SDIO{\_}CMD为命令信号,SDIO{\_}CK为时钟信号,SDIO{\_}D[3:0]为数据信号。为降低功耗,\acrshort{WiFi}模块也有掉电控制引脚PDn,低电平时模块进入掉电模式,耗电极低,高电平时模块进入工作模式。由STM32微控制器的PB5引脚控制\acrshort{WiFi}模块掉电或工作。
\pic[htbp]{\acrshort{WiFi}模块电路原理图}{width=0.8\textwidth}{wifi.pdf}

\section{呼叫按钮与掉电管理电路}\label{sec:butn_pdwn}
考虑到病房语音呼叫的特殊性,系统硬件一般情况下使用外部直流电源供电,遇到停电等特殊情况时使用内部电池作为后备电源。为尽量降低功耗,延长电池寿命,使用PCF8563作为定时掉电管理芯片。常态下系统电源关闭,在三种情况下系统会被唤醒:(1)PCF8563经过程序预定的时间后输出一唤醒信号(类似于关机闹钟);(2)语音呼叫按钮被按下;(3)防破坏拆除按钮被断开。任何一种情况都会使系统电源开启,STM32微控制器上电。STM32开始工作后首先判断唤醒原因并执行不同的程序:(1)定时唤醒,使能\acrshort{WiFi}模块,尝试向服务器发送心跳数据包表明自己工作正常并判断网络是否通畅;(2)语音呼叫按钮S2被按下,定时唤醒,使能\acrshort{WiFi}模块,尝试向服务器发送语音呼叫请求;(3)防破坏拆除按钮S1被断开,说明设备被人为破坏,警报音响起,同时使能\acrshort{WiFi}模块,尝试向服务器发送报警数据包。
\pic[htbp]{呼叫按钮与掉电管理电路原理图}{width=0.6\textwidth}{pcf8563.pdf}

\section{系统硬件的PCB版图}
系统硬件的原理图和PCB版图均使用Altium Designer设计,PCB三维效果图如图~\ref{pics_sysPCB}所示。PCB尺寸65 mm $\times$ 56 mm,外壳使用标准“八六盒”(86 mm $\times$ 86 mm)。
\begin{pics}[htbp]{系统硬件的PCB版图}{pics_sysPCB}
\addsubpic{Toplayer}{width=0.45\textwidth}{toplayer.png}
\addsubpic{Bottomlayer}{width=0.45\textwidth}{bottomlayer.png}
\end{pics}

\section{本章小结}
本章主要阐述了系统硬件的总体设计方案,对系统电源、微控制器、语音输入输出电路、\acrshort{WiFi}模块电路等各个功能的硬件电路进行了分别详细的阐述,并给出了对应的电路原理图。