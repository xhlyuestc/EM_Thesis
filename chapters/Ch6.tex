% !Mode:: "TeX:UTF-8"

\chapter{全文总结与展望}
\section{本文的主要贡献}
本论文主要论述了医院住院病房语音呼叫系统的研究与开发工作,系统运行效果来看,系统的各个功能模块基本达到了预期的要求,实现了医院语音呼叫系统。主要完成了以下工作:
\begin{enumerate}
\item 详细分析了本系统的需求,编写了详细的需求分析文档。
\item 采用B/S系统体系结构。采用三层架构模式,充分利用网络的自身优势,将系统部署网络上进行应用;设置专门的应用服务器和数据服务器保证系统的正常运行,保证系统大规模使用时能很好的提供服务;重新设计的三层架构模式对系统的数据传输和数据的处理有很大的提高,提升了医院住院语音系统的模式。
\item 使用SQL server 2005建立起系统需要的数据库和数据表,配合JSP所提供的验证的内建数据表,并利用了ASP所内建的SqlMembershipProvider和RoleProvider模型来简化对成员与角色的管理。
\item 浏览器模式可以更加方便的进行访问,减少了系统的维护费用,方便了系统的推广以及使用。在数据传输和数据库共享上,相比客户端模式有着较大的改善。系统数据的安全性以及语音呼叫工能都得到了很大的改善,对系统的数据分析以及病房管理等功能都进行了改进。 
\end{enumerate}

\section{存在的不足}
本文主要对医院住院语音呼叫系统进行了设计和实现,实现了系统的各个功能模块以及系统的数据分析和数据输出功能,展示了系统的各个功能模块,但是对系统的参数设置以及系统的数据分析预测功能还有待完善,对不同的数据分析还有待提高,对系统的可扩展性以及系统的稳定性实时性等都有待提高。

\section{展望}
随着单片机技术和通信技术的发展,病房呼叫系统的功能也越来越多,越来越完善,朝着人性化、智能化、网络化的方向发展。

虽然本文设计实现的病房语音呼叫系统仍然存在一定的不足,相信经过更多的优化和完善,以及其他学者的进一步探索,医院病房的语音呼叫和管理系统并将越来越完善而智能化。相信在不久,病房呼叫系统会与医院信息系统联合起来,在发挥更重要的作用。