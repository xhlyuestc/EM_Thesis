% !Mode:: "TeX:UTF-8"

\chapter{绪论}
\section{研究背景}
随着计算机通讯技术的飞速发展,基于计算机技术和通信技术的新型病房呼叫系统应运而生,其应用可以更有效地帮助医护人员及时掌握患者的突发急危病情,尤其是无人陪护的急病患者的准确呼救信息,对迅速到达现场实施抢救提供了技术保障;同时它也是现代化医院护理,医院医疗管理体系的重要组成部分。

当前市场上存在着许多种型号不一功能各异的医院病房呼叫系统,主要为两大类:有线式和无线式。传统的有线式病房呼叫系统往往采用集中式结构,电源线、数据通信线、语音通信线分开传输,铺设线路较多、成本高、安装调试困难、实时性差、故障率较高等缺点;而无线式病房呼叫系统不存在铺设线路的问题,但是可靠性差,而且无线电波会干扰其它医疗仪器设备,目前大多数医院不采用此类无线呼叫系统。以此,本课题以研究并设计一种简单实用、安全可靠、性能稳定具有良好性价比的病房呼叫系统,对于我国基层医院的现代化建设有十分重要的意义。本课题的研究将以呼叫系统的稳定性、语音数据传递的准确性、设备的复杂性等方面的问题作为主要研究内容。

病房呼叫系统的发展大概可以分为3个阶段,即传统上的口头呼叫、摇铃呼叫和电子按铃呼叫3个阶段。相比而言,国外医院的呼叫系统发展比较迅速,现在己经渐渐出现了朝着护理系统的智能化和可视化发展,而我国的病房呼叫系统起步较晚,还处在电子按铃呼叫的初级阶段,与多年前相比变化不大。病床呼叫系统一般具有声光提示功能,有的系统有通话功能,使医护人员能够了解病人的医护请求。除人性化的辅助功能外,研究重点主要集中在主机与病房呼叫终端之间通信的安全性、可靠性、施工、系统构建、可扩展性等技术问题上进行研究。其通信方式可归纳为:有线通信和无线通信两种方式。

有线通信医疗呼叫系统,主机与病房呼叫终端之间采用导线连接进行通信和供电。20世纪80年代初,医疗呼叫一般都采用内部电话来构成呼叫系统,技术成熟,这种方式优点是:可进行双向呼叫和通话功能。缺点是:每个病床需要连接一根电话线,布线施工麻烦,成本也高。随着总线通信技术的发展和单片机应用的普及,采用CAN总线、RS-485总线、XY•CN总线、二线制或四线制脉冲编码通信总线、电力载波等总线通信技术的医疗呼叫系统得到了广泛应用。无线通信医疗呼叫系统,主机与病房呼叫终端之间采用无线电波或红外光等无线通信手段进行通信,呼叫终端一般采用电池供电。主机与病房呼叫终端之间不需要布线,具有安装施工简单,能满足移动呼叫的需求。 

随着我国医疗行业进入一个以服务为核心的新的发展周期,国内众多学者对于新型呼叫系统的设计与推广进行了广泛地研究。病床呼叫系统是病人请求值班医生或护士进行诊断或护理的紧急呼叫工具。可将病人的请求快速传送给值班医生或护士,是提高医院和病室护理水平的必备设备之一。如何利用先进的信息技术为医院服务,更大程度的提高医院的服务质量及利润,是医院信息化建设中的一个重要着眼点。随着现在计算机技术不断的发展,迫切的希望研发一套医院病房呼叫系统,来应对病房病人的需求。

\subsection{国内外发展现状}
国外医院的呼叫系统发展比较迅速,现在己经渐渐出现了朝着护理系统的智能化和可视化发展,而我国的病房呼叫系统起步较晚,还处在电子按铃呼叫的初级阶段,与多年前相比变化不大。

国内的多位学者对医院病房呼叫系统都做出了研究曾进辉在《基于DTMF的医院护理呼叫系统的设计与实现》一文中提出了系统以单片机AT89C52为核心,采用信号发送芯片、信号接受芯片、语音播报芯片、发光二极管、数码管显示等外围电路以及相应的控制程序,实现了通过电话拨号进行单呼、群呼、显示、呼叫提示、查询、播报、对讲及护理级别的设置和删除等功能通过编码、传输及解码技术,能确保数据远距离传输且抗干扰能力强,避免了有线寻呼系统传输的不稳定性。

朱艳华、田行军、李夏青等在《基于PL3105的病房呼叫系统设计》一文中提出了针对传统病房呼叫系统存在扩展困难或无线电干扰的现状,设计了基于可编程载波通信芯片PL3105的新型病房呼叫系统。系统分控制器和载波终端两层结构,均以PL3105为主控制单元,并以低压电力线为“总线”传输相应的控制命令。系统软件采用功能模块化设计思想,提高了系统移植性和可靠性。实测证明,新型病房呼叫系统不仅满足医疗单位的功能需求,而且具有扩展性强、维护方便、经济实用等优点,有较强的推广价值。

乔国鹏在《基于ARM的数字化病房呼叫系统》一文中详细介绍了基于ARM的数字化病房呼叫系统的设计。以微控制器STM3210为核心控制终端设备接收和发送,采用RS232、RS485及UDP通信技术,实现了终端设备与服务台之间语音和通讯命令传输。该系统在实际使用中效果良好。系统利用单片机的自动控制特性,使得系统稳定、可靠。系统采用的元器件均是常见的电子元器件,因此系统硬件成本较低。分机具有较低的功耗,并且具有较好的扩展性。主机与分机的通信稳定,实时性好,能满足各种规模医院的要求,有很好的应用前景。

郭广颂、胡璞在《基于单片机的无线病房呼叫系统设计》一文中详细介绍了在传统病房呼叫系统的基础上提出一种基于单片机实现的无线病房呼叫系统设计方法给出了基于单片机和射频通信芯片nRF401而设计的硬件原理图,并对影响无线通信性能的通信协议的设计、数据帧和防信息碰撞方法的实现及混合信号PCB板设计等做了较详细的探讨。

潘绍明、梁喜幸《基于信号叠加和无线电的病房呼叫系统设计与实现》一文中介绍了包括走廊主机、监控室主机、电脑上位机、手持式监控机和病床呼叫分机的病房呼叫系统。各床位的呼叫信号通过两根既作为信号传输又为各病床呼叫分机提供电源的电线发送到走廊主机,再由走廊主机以无线电的方式向监控室主机和手持式监控机发送。监控室主机和手持式监控机在接收到信号后将会做出相应处理和反应。本设计加入了无线电通信,提高了医务人员工作的灵活性,能在无线电覆盖范围的任何一个位置接收到病人的呼叫信号,保证了病人的呼叫信号能在第一时间得到响应,很大程度上保障了病人的身体健康和生命安全。

通过对上述研究成果的综合分析,和对目前医疗呼叫系统的工程实现的经验可以把当前医疗呼叫系统产品中存在的不足,归纳为以下几类:
\begin{enumerate}
   \item 有线通信医疗呼叫系统:病房和值班室之间需要大量的连线,安装布线复杂,检查维修困难,病房扩建不易及费用高,不固定应用场所使用不便。
   \item 单工无线通信医疗呼叫系统:可靠性低,抗干扰能力差,发射功率大,对医疗设备有干扰。
   \item ZigBee协议无线通信医疗呼叫系统:载波频率高,穿透墙壁能力差,通信距离短;支持ZigBee协议的无线模块和微控制器成本相对较高。
\end{enumerate}

近年来设计生产的呼叫器已普遍采用了单片机,使其功能大为增强。同时,值班室与病房间的连线也大大减少,布线简洁方便,迎合了医院追求环境整洁的需求。即便如此,但还是无法摆脱电线的束缚,布线麻烦,遇到病房扩建或改造,系统则需要重新布线,产品的重复使用率低,致使成本增加,需要新一代的产品来改善。目前市售的各种呼叫器均不具备个性录音功能,而在临床实践中发现,好多的医院需要自己独特的录音内容,这对于医生的护理将有大好处,这也是本文所介绍的病房语音呼叫系统功能上的创新点之一。

\section{研究内容和意义}
伴随着医疗体制改革的不断深化和医疗事业的飞速发展,越来越多的人们需要迅捷、方便地得到医院的各种各样的医疗服务,这必将使医院之间的竞争日趋激烈。这使得衡量一个医院的综合水平高低,不再仅仅局限于软、硬件的建设上,更要比服务。原有的服务体系已不足以适应现代社会需求,谋求适合现代社会需求的客户服务系统,是所有企事业单位计划做的工作。这些工作有利于改善服务量,提高效率并增加企业效益,从而赢得良好的社会声誉。如何利用先进的信息技术为医院服务,更大程度的提高医院的服务质量及利润,是医院信息化建设中的一个重要着眼点。

医院的竞争越来越激烈,商业医院的生存是第一位的,提高档次和服务质量迫在眉睫,陪护问题一直是医患矛盾的主体,也是长期困扰卫生系统服务质量的大问题,使用无线呼叫系统,方便病人更快找到医生,以节约病人的宝贵时间。临床呼叫求助装置是传送临床信息的重要手段,关系病员安危,传统的有限呼叫系统历来受到各大医院的普遍重视。如果采用无线传输,会节约布线和改造线路的资金,为医院节约成本,并且及时、准确、可靠,简便可行,必然比目前的同类产品更能受到医院及病人的认可,有更强的竞争力,必然能大力推广。

本文主要涉及硬件设计、通信技术、软件工程等的基本理论。研究内容包括语音呼叫终端的供电、主控电路、通信接口和通信协议、语音采集与回放、数据存储于访问等。

临床呼叫装置是传送临床信息的重要手段,关系患者的安危。呼叫系统历来受到各大医院的普遍重视。为了方便患者,提高医院服务质量,医用呼叫系统已经成为了国内外各类医院中广泛使用的一种电子设备。它能从根本上解决传统医患之间所存在的一些服务纠纷等问题,可以静化医院的工作环境,避免无谓的争执。既可以帮助病人快速的呼叫医护人员,也可减轻医护人员巡视病房的辛劳,减轻医护人员值班的心理压力,在无呼叫时放心的作好其他医护工作,从而提高了医护效率。因此,医用呼叫系统具有广泛的社会意义与重大的实用价值。

\section{研究思路和方法}
本课题主要采用文献研究、需求调研与分析、总体方案论证、硬件设计和软件设计等方法,根据EDA/CAD、嵌入式系统和面向对象的程序设计等技术进行指导设计与测试。

呼叫终端的硬件设计是整个病房呼叫系统设计的基础,在系统设计中占有非常重要的作用。而它的硬件设计中,最重要的组成部分就是主控处理器模块,主控处理器模块直接影响硬件终端的功能、性能优劣。因此,硬件方案的选择,其实就是终端硬件主控处理器模块和其他外设的确定,其决定了整个系统功能的优劣。

本课题涉及两个部分的软件开发过程,上位机部分和下位机部分,下位机为单片机部分,是指病患和医生使用的终端器件,其软件开发是基于模块化的功能实现,需要针对医生和病人设计不同的终端呼叫设备,上位机部分为电脑上运行的软件,或主控部分,或服务器部分,应该针对市面主流操作系统进行开发,完成对整个系统呼叫要求的申请和处理。

在通信方面,本系统设计时主要涉及物理层和媒体接入协议。当前市面上还没有为医院呼叫系统开发出专用的媒体接入协议,但是可以借鉴无线局域网的MAC协议的部分内容。针对静止或者移动缓慢的呼叫终端,可以通过有线网接入服务器,针对需要移动的终端可以选择带无线局域网的接入方式,可以考虑通过中心节点来控制局部网络,形成小型的多层局域网,或者选择微波中继站,或者ZigBee协议,来提高整个通信的一条距离,来减少整个系统的设计难度,当前最通用的无线局域网媒体接入协议可以根据硬件选择IEEE802.11、IEEE802.11b、IEEE802.11a、IEEE802.11g、IEEE802.11n等多种无线协议标准。

课题研究中,将使用Altium Designer设计系统的硬件电路原理图和PCB版图,拟采用STM32单片机作为语音呼叫终端的微控制器,配合国内开源的RT-Thread嵌入式实时操作系统管理所有硬件设备并完成语音的网络传输。嵌入式软件开发工具选择Keil uVision/MDK-ARM,上位机软件采用Microsoft的Visual Studio/C\# ,数据库管理采用SQL。

\section{本文的结构安排}
第一章,绪论,本章主要对系统研究的背景以及系统研究的国内外发展现状作了详细的分析,对本文的研究意义以及本文所做的工作做了详细的分析。

第二章,语音呼叫系统的相关技术研究,本章对系统所用的到相关技术作了详细的介绍,对系统设计的软硬件技术以及系统的传输协议等进行了论述。

第三章,系统的硬件设计与实现,本章对系统的下位机硬件进行了详细的设计,实现了下位机的通信,通过对通信协议的设计优化保证了整个系统的通信可靠性。

第四章,系统的软件设计与实现,对上位机和下位机系统主要功能模块的设计和实现进行了详细阐述,并对上下位机进行联合调试,实现了整个系统的所有功能模块。

第五章,系统的部署与联合测试,对整个系统进行部署,对系统部署所需要的软硬件的环境进行了论述,以及系统的整个功能进行了测试,测试了系统功能的可用性以及系统的稳定性。

第六章,总结与展望,总结了本文的工作,对系统发展的前景作了论述。

本文提出了一种医院病房语音呼叫系统的解决方案,方便了医院对病房和病人的管理,解决了人工巡视病房的缺点,让病人随时可以方便地呼叫医护人员。系统硬件上采用无线传输以及微控制器电路设计,保证了系统的简洁、灵活、稳定、功耗低,底层硬件和顶层上位机系统通过无线网络实现互联,方便了系统的使用与部署。
